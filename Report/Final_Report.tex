\documentclass[10pt,a4paper]{article}
\usepackage[utf8]{inputenc}
\usepackage{amsmath}
\usepackage{amsfonts}
\usepackage{amssymb}
\newcommand\tab[1][1cm]{\hspace*{#1}}
\usepackage{graphicx}
\usepackage{scrextend}
\usepackage{wrapfig}
\author{Konrad Cybulski, Julian Kardis, Matteo Batelic}
\title{Understanding Trends in Global Terrorism}
\begin{document}
\begin{titlepage}
    \begin{center}
        \vspace*{1cm}
        
        \LARGE
        \textbf{Understanding Trends in Global Terrorism}
        
        \vspace{4cm}
        
		\Large 
        
        \textbf{Konrad Cybulski, Julian Kardis, Matteo Batelic}
        
        
        \LARGE
        \vspace{2cm}

        
        
        \vfill
        
        
        
        Final report \\
        FIT2083 Research Project
        
        
        \includegraphics[width=0.4\textwidth]{monash-university-logo.png}
              
        
        \large
        Faculty of Information Technology\\
        Monash University\\
        Australia\\
        24/10/2016
        
    \end{center}
\end{titlepage}

\pagebreak
\tableofcontents
\pagebreak


\section{Abstract} 

\section{Introduction} 

\section{Background} 

\section{Method}
\subsection{Data}
The database used in this investigation comes from the Global Terrorism Database (GTD) (START, 2016) with 156,749 listings of successful and failed terrorist attacks around the world between 1970 and 2015. With additionally 137 variables including date, country, latitude/longitude, number of perpetrators, number of deaths as a result of the attack, number of injuries, method of attack (bombing, armed assault, etc.) as well as many others. Due to the extensive nature of the database, this investigation aims to understand trends in a select number of these variables. These variables include attack types, deaths, number of attacks, number of successful attacks, and country of attack. In order to understand country specific and global trends in terrorism, the data will be explored not only as a whole, but as a series of subsets of the GTD grouped by country as well as a series of country clusters.
\\\\

\subsection{Analysis}
The methods used to analyse the GTD involved for the most part an analysis of a number of time series'. In order to understand trends of not only single countries, but groups of countries as well as terrorism on a global scale, the groups and countries used were determined in the following way. The country clusters investigated were made up of three groups, typically Western and first world countries (United States, Canada, Australia, France, Finland, Russia, etc.), the six countries with the overall highest number of deaths due to terrorist activity between 1970 and 2015 (which is determined by data from the GTD), and a more in depth look solely at trends in the United States.
\\\\
All three groups are investigated with regard to deaths due to terrorist activity, number of attacks, type of attack and rate of success for a given attack. However the United States is not only subject to investigation into these variables, but using data from the \textit{New York Times}, investigated with regard to possible relationships between the media's coverage of terrorism and rates of attacks and methods of attacks.


\section{Results}
The results produced as outlined by the \textit{Methods} section are below grouped by the variable which 


\section{Discussion} 

\subsection{Terrorism coverage in United States news data}

\subsection{Methods of Attack}

\subsection{Terrorism trends over time}

\subsection{Rates of success in terrorist attacks}

\section{Conclusion} 








\pagebreak
\begin{thebibliography}{9}

\bibitem{2}
National Consortium for the Study of Terrorism and Responses to Terrorism (START). 
(2016). \textit{Global Terrorism Database}[Data file].
Retrieved from https://www.start.umd.edu/gtd

\end{thebibliography}


\end{document}